%%%%%%%%%%%%%%%%%%%%%%%%%%%%%%%%%%%%%%%%%%%%%%%%%%%%%%%%%%%%%%%%%%%%%%%%%%%%%%%%%%
\begin{frame}[fragile]\frametitle{}
\begin{center}
{\Large Introduction to YogaNidra}
\end{center}
\end{frame}

%%%%%%%%%%%%%%%%%%%%%%%%%%%%%%%%%%%%%%%%%%%%%%%%%%%%%%%%%%%
\begin{frame}[fragile]\frametitle{What is Yoganidra?}
	\begin{itemize}
	\item a very long Savasana
	\item is the practice of psychic sleep, a conscious state that flirts on the border of sleep.
	\item can be practiced at any time, except right after eating, as you may be more inclined to fall asleep then.
	\end{itemize}

\end{frame}

%%%%%%%%%%%%%%%%%%%%%%%%%%%%%%%%%%%%%%%%%%%%%%%%%%%%%%%%%%%
\begin{frame}[fragile]\frametitle{Preparation}
	\begin{itemize}
	\item Lie down on your back 
	\item Bring your feet as wide as your mat. 
	\item Rest your arms by your sides, away from your body, and palms facing the sky. 
	\item Get completely comfortable. Adjust yourself so that you are comfortable on the floor. 
	\item Close your eyes. 
	\item Now find stillness. 
	\item Do not move throughout the practice of yoga nidra. 
	\item Keep your body still and relaxed. 
	\item Release any tension from your body and find complete relaxation. 
	\item Stay awake throughout the practice of yoga nidra. 
	\item Tell yourself, "I will not sleep. I will remain awake." 
	\item Simply follow guiding voice throughout the practice but do not try to intellectualize my words. 
	\end{itemize}

\end{frame}

%%%%%%%%%%%%%%%%%%%%%%%%%%%%%%%%%%%%%%%%%%%%%%%%%%%%%%%%%%%
\begin{frame}[fragile]\frametitle{Steps}
The 7 Stages of Yoga Nidra:
	\begin{itemize}
	\item Make a resolve or affirmation.
	\item Rotate your consciousness throughout the annamaya kosha (physical body).
	\item Bring attention to the breath to connect to the pranamaya kosha (pranic body).
	\item Become aware of sensory input to tap into the manomaya kosha (mind body).
	\item Direct your attention to the chakras and psychic symbols to connect to the vijnanamaya kosha (intellectual body).
	\item Repeat your resolve to connect to the anandamaya kosha (bliss body).
	\item Gradually return to your normal level of consciousness, concluding the practice.
	\end{itemize}

\end{frame}

% %%%%%%%%%%%%%%%%%%%%%%%%%%%%%%%%%%%%%%%%%%%%%%%%%%%%%%%%%%%
% \begin{frame}[fragile]\frametitle{Affirmation}
	% \begin{itemize}
	% \item Not just positive thinking but a resolve for your betterment
	% \item A short sentence in present-tense that you want and believe will surely come true.
	% \item It should be clear, direct, and simple. 
	% \item Fix your resolve and use this resolve now and at the end of the yoga nidra practice.
	% \item E.g. "my body is healthy", "I am lovable"
	% \item Our resolve settles into the subconscious through the practice of yognidra, enabling deep and profound changes in your self-perception and the body.
	% \end{itemize}

% \end{frame}

% %%%%%%%%%%%%%%%%%%%%%%%%%%%%%%%%%%%%%%%%%%%%%%%%%%%%%%%%%%%
% \begin{frame}[fragile]\frametitle{Traversing through Body}
	% \begin{itemize}
	% \item direct your awareness where you next hear the voice
	% \item After the voice says a body part, mentally repeat the name of that body part. 
	% \item Right thumb, second finger, third finger, fourth finger, fifth finger, wrist, elbow, shoulder, waist, hip, thigh, knee, calf muscle, ankle, heel, sole of the foot, right big toe, second toe, third toe, fourth toe, fifth toe. 
	% \item Then repeat on the left side.
	% \item Feel your body get very heavy. 
	% \item Awaken the feeling of heaviness throughout your body. 
	% \item Become aware of your whole body heavy on the floor. 
	% \item Now awaken the feeling of lightness. Feel your body being very light. 
	% \item Feel each part of the body very light. 
	% \item Repeat to yourself, "I am not sleeping."
	% \end{itemize}

% \end{frame}



%%%%%%%%%%%%%%%%%%%%%%%%%%%%%%%%%%%%%%%%%%%%%%%%%%%%%%%%%%%
\begin{frame}[fragile]\frametitle{General steps of a Yoga Nidra practice}
	\begin{itemize}
	\item Preparation:
	\begin{itemize}
		\item Lie down on your back in a comfortable position, with your legs slightly apart and arms relaxed by your sides.
		\item Close your eyes and take a few deep breaths to relax your body and mind.
	\end{itemize}
		
	\item Set an Intention:
	\begin{itemize}
		\item Reflect on your intention or resolve for the practice. This could be a specific goal or an affirmation that you repeat to yourself.
	\end{itemize}
		
	\item Body Awareness:
	\begin{itemize}
		\item Bring your attention to different parts of your body, starting from the right side.
		\item Slowly move your awareness through each body part, such as the right hand, right forearm, upper arm, shoulder, and so on.
		\item Feel the sensation or heaviness in each body part and let go of any tension or discomfort.
	\end{itemize}
		
	\item Breath Awareness:
	\begin{itemize}
		\item Shift your attention to your breath.
		\item Observe the natural flow of your breath without trying to control it.
		\item Notice the sensation of the breath as it enters and leaves your body.
	\end{itemize}
		
	\item Sensory Awareness:
	\begin{itemize}
		\item Bring your awareness to your senses, one at a time.
		\item Mentally repeat statements like "I am aware of sounds" or "I am aware of sensations on my skin."
		\item Allow your awareness to move from one sense to another without getting attached to any particular sensation.
	\end{itemize}		
	\end{itemize}

\end{frame}


%%%%%%%%%%%%%%%%%%%%%%%%%%%%%%%%%%%%%%%%%%%%%%%%%%%%%%%%%%%
\begin{frame}[fragile]\frametitle{General steps of a Yoga Nidra practice}
	\begin{itemize}

		
	\item Visualization:
	\begin{itemize}
		\item Imagine or visualize various objects or scenes as guided by the instructions or the voice guiding the practice.
		\item Engage your senses in the visualization, incorporating details like colors, textures, and smells.
	\end{itemize}
		
	\item Repeat the Sankalpa:
	\begin{itemize}
		\item Return to your initial intention or resolve.
		\item Repeat your intention silently in your mind three times with conviction and belief.
	\end{itemize}
		
	\item Gradual Return:
	\begin{itemize}
		\item Slowly bring your awareness back to your physical body and the surroundings.
		\item Gently wiggle your fingers and toes, stretch your body, and take a few deep breaths.
	\end{itemize}
		
	\item Reflect:
	\begin{itemize}
		\item Take a moment to reflect on your experience during the practice.
		\item Notice any shifts in your mental or emotional state.
	\end{itemize}
		
	\end{itemize}

\end{frame}

%%%%%%%%%%%%%%%%%%%%%%%%%%%%%%%%%%%%%%%%%%%%%%%%%%%%%%%%%%%
\begin{frame}[fragile]\frametitle{Benefits}
	\begin{itemize}
	\item Brings deep relaxation
	\item Withdraws the senses (pratyahara)
	\item Resolves bad habits
	\item Releases body tension
	\item Helps you fall asleep
	\item Brings connection to the inner self
	\item Taps into the subconscious
	\item Can aid in physical healing
	\end{itemize}

\end{frame}

%%%%%%%%%%%%%%%%%%%%%%%%%%%%%%%%%%%%%%%%%%%%%%%%%%%%%%%%%%%
\begin{frame}[fragile]\frametitle{References}
	\begin{itemize}
	\item Yoga Nidra 101: The Practice of Psychic Sleep - beYogi https://beyogi.com/yoga-nidra-practice-psychic-sleep/
	\end{itemize}

\end{frame}

